Motion trajectories are really difficult to analyse and handle because of the amount of data. There could be errors in position tracking and usually you don't have informations like the city from which the positions are taken, or any higher semantic level information that can help in trajectories grouping. Consequently trajectories are difficult to represent, filter and manage in relation with itself or other data. First of all, trajectories clustering is a challenge for its intrinsic difficulty in being treated and today is an ambitious topic in data science research. Moreover trajectory clustering can help for several applications:
\begin{itemize}
	\item Monitoring: understand main points of interest and common places visited by tourists;
	\item Forecasting: prediction of possible destinations starting from a position; 
	\item Viability: traffic monitoring and detection of possible point of traffic jam;
\end{itemize}
